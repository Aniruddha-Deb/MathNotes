\documentclass[12pt]{article}
\usepackage{amsmath}
\usepackage{lipsum}
\usepackage{mathtools}
\usepackage{titling}
\usepackage[paperwidth=25cm,paperheight=50cm,margin=0.3in,footskip=0.25in]{geometry}

%opening
\pretitle{\vspace{-2cm}}
\title{}
\posttitle{}
\preauthor{}
\author{}
\postauthor{}
\predate{}
\date{}
\postdate{}

\setlength{\jot}{8pt}
\pagenumbering{gobble}

\newcommand{\ppmm}{\begin{array}{c|c}+&+\\\hline\\[-12pt]-&-\end{array}}
\newcommand{\pmpm}{\begin{array}{c|c}-&+\\\hline\\[-12pt]+&-\end{array}}
\newcommand{\pmmp}{\begin{array}{c|c}-&+\\\hline\\[-12pt]-&+\end{array}}
\DeclareMathOperator{\cosec}{cosec}

\newcommand{\pwidth}{0.46\linewidth}

\begin{document}
	
	\maketitle
	
	\section{General Trigonometric Formulae}
	
	\begin{minipage}[t]{\pwidth}
		\subsection{Addition Formulae}
		\begin{align}
			\sin(A \pm B) &= \sin A \cos B \pm \cos A \sin B \\
			\cos(A \pm B) &= \cos A \cos B \mp \sin A \sin B \\
			\tan(A \pm B) &= \frac{\tan A \pm \tan B}{1 \mp \tan A \tan B} \\
			\cot(A \pm B) &= \frac{\cot B \cot A \mp 1}{\cot B \pm \cot A}
		\end{align}
		
		\subsection{Double Angle Formulae}
		\begin{align}
			\sin 2A &= 2 \sin A \cos A = \frac{2 \tan A}{1 + \tan^2A} \\
			\cos 2A &= \cos^2A - \sin^2A \nonumber \\ &= 2\cos^2A - 1 \nonumber\\ &= 1 - 2\sin^2A \nonumber\\ &= \frac{1-\tan^2A}{1 + \tan^2A} \\
			\tan 2A &= \frac{2\tan A}{1- \tan^2A}
		\end{align}
		
		\subsection{Half Angle Formulae}
		\begin{align}
			\sin \frac A2 &= \pm \sqrt{\frac{1-\cos A}{2}} \qquad \ppmm \\
			\cos \frac A2 &= \pm \sqrt{\frac{1+\cos A}{2}} \qquad \pmmp \\
			\tan \frac A2 &= \pm \sqrt{\frac{1-\cos A}{1+\cos A}} \qquad \pmpm \nonumber\\	&= \frac{\sin A}{1+\cos A} \nonumber\\ &= \frac{1-\cos A}{\sin A} \nonumber\\ &= \cosec A - \cot A
		\end{align}

		\subsection{Product to Sum Formulae}
		\begin{align}
			2\sin A \sin B &= \cos(A-B) - \cos(A+B) \\
			2\cos A \cos B &= \cos(A-B) + \cos(A+B) \\
			2\sin A \cos B &= \sin(A-B) + \sin(A+B)
		\end{align}
	\end{minipage}%
	\hfill
	\begin{minipage}[t]{\pwidth}
		\subsection{Multiple Angle Formulae}
		\begin{align}
			\sin 3A &= 3\sin A - 4\sin^3A \\
			\cos 3A &= 4\cos^3A - 3\cos A \\
			\tan 3A &= \frac{3\tan A - \tan^3A}{1 - 3\tan^2A}
		\end{align}
		\begin{align}
			\tan\left(\displaystyle \sum_{i=1}^n a_i\right) = \frac{e_1 - e_3 + e_5 + ...}{1 - e_2 + e_4 + ...}
		\end{align}
		where \vspace{-10pt}
		\begin{align}
			e_1 &= \displaystyle \sum_{i=1}^n a_i \nonumber\\
			e_2 &= \displaystyle \sum_{i < j} a_i a_j \nonumber\\
			e_3 &= \displaystyle \sum_{i < j < k} a_i a_j a+k \nonumber
		\end{align}
		and so on.

		\subsection{Sum to Product Formulae}
		\begin{align}
			\sin C + \sin D &= 2\sin\left(\frac{C+D}2\right)\cos\left(\frac{C-D}2\right) \\
			\sin C - \sin D &= 2\sin\left(\frac{C-D}2\right)\cos\left(\frac{C+D}2\right) \\
			\cos C + \cos D &= 2\cos\left(\frac{C+D}2\right)\cos\left(\frac{C-D}2\right) \\
			\cos C - \cos D &= -2\sin\left(\frac{C+D}2\right)\sin\left(\frac{C-D}2\right)
		\end{align}

		\subsection{Difference of Squares}
		\begin{align}
			\sin^2A - \sin^2B &= \sin(A-B)\sin(A+B) \\
			\cos^2A - \sin^2B &= \cos(A-B)\cos(A+B)
		\end{align}

		\subsection{Special Results}
		\begin{align}
			\sin(A)\sin\left(\frac{\pi}{3} - A\right)\sin\left(\frac{\pi}{3} + A\right) = \frac 14 \sin 3A \\
			\cos(A)\cos\left(\frac{\pi}{3} - A\right)\cos\left(\frac{\pi}{3} + A\right) = \frac 14 \cos 3A \\
			\tan(A)\tan\left(\frac{\pi}{3} - A\right)\tan\left(\frac{\pi}{3} + A\right) = \tan 3A
		\end{align}
	\end{minipage}

	\section{Sum and Product of Trigonometric Functions}
	\subsection{Product formulae}
	\begin{align}
		C &= \cos \theta \cos 2\theta \cos 4\theta \cdots \cos 2^{n-1}\theta \nonumber \\
		  &= \frac{\sin\theta}{\sin\theta}\cos \theta \cos 2\theta \cos 4\theta \cdots \cos 2^{n-1}\theta \nonumber \\
		  &= \frac{\sin 2\theta}{2 \sin\theta}\cos 2\theta \cos 4\theta \cdots \cos 2^{n-1}\theta \nonumber \\
		C &= \frac{\sin 2^n\theta}{2^n \sin\theta} \\
		\nonumber \\
		S &= \cos\frac{\theta}{2} \cos\frac{\theta}{4} \cos\frac{\theta}{8} \cdots \cos\frac{\theta}{2^n} \nonumber \\
		  &= \frac{1}{\sin\frac{\theta}{2^n}} \cos\frac{\theta}{2} \cos\frac{\theta}{4} \cos\frac{\theta}{8} \cdots \cos\frac{\theta}{2^n} \sin\frac{\theta}{2^n} \nonumber \\
		S &= \frac{1}{2^n\sin\frac{\theta}{2^n}} \sin\theta \\
		S &\to \frac{\sin\theta}{\theta} \text{ as } n \to \infty
	\end{align}

	\subsection{Sum formulae}
	\begin{align}
		\sum_{r=1}^{n} \sin(\alpha + (r-1)\beta) = \frac{\sin\frac{n\beta}{2}}{\sin\frac{\beta}{2}} \sin\left( \alpha + (n-1)\frac\beta 2\right) \\
		\sum_{r=1}^{n} \cos(\alpha + (r-1)\beta) = \frac{\sin\frac{n\beta}{2}}{\sin\frac{\beta}{2}} \cos\left( \alpha + (n-1)\frac\beta 2\right)		
	\end{align}
\end{document}